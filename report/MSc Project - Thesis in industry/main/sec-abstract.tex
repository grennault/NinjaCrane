\begin{abstract}
Despite the sophisticated \emph{Stuxnet} attack discovered in 2010, the cybersecurity awareness around the \emph{Industrial and Control System} (ICS) remains a challenge~\cite{Pattanayak18}. The necessity to train ICS actors and to raise awareness on this issue increases~\cite{Wang22}~\cite{wong22}~\cite{cisaTraining}. Moreover, listing the cybersecurity best practices to follow regarding a security policy has a limited impact on their effectiveness compared to an awareness program~\cite{Rhee09}~\cite{Sitnikova13}~\cite{Angafor20}. In this context, this Master’s Project builds a cyberattack demonstrator targeting a part of a nuclear power-plant ICS and more precisely a polar crane's \emph{control system}. The described attack, called \emph{NinjaCrane}, infects an engineering workstation through an USB cable or a maliciously tampered mouse and deploys a malware targeting the communication with a \emph{Modicon M580}, a \emph{Programmable Logic Controller} (PLC), by conducting a \emph{Meet-In-The-Middle Attack} (MITM) exploiting the CVE-2022-45789 (Capture-replay attack) and the CVE-2021-22779 vulnerability (a.k.a. \emph{ModiPwn}). This work illustrates and highlights the necessity to respect the ICS security policy and may well suit an awareness program.

\iffalse
The \sysname tool enables lateral decomposition of a multi-dimensional flux compensator along the timing and space axes.

The abstract serves as an executive summary of your project.
Your abstract should cover at least the following topics, 1-2 sentences for each: what area you are in, the problem you focus on, why existing work is insufficient, what the high-level intuition of your work is, maybe a neat design or implementation decision, and key results of your evaluation.
\fi
\end{abstract}

\begin{frenchabstract}
Malgré la sophistication de l'attaque \emph{Stuxnet}, découverte en 2010, la prise de conscience sur l'importance de la cybersecurité des systèmes de supervision reste encore en marge~\cite{Pattanayak18} et la necessité de former les acteurs industriels sur cette problématique apparaît comme croissante~\cite{Wang22}~\cite{wong22}~\cite{cisaTraining}. De plus, lister les bonnes pratiques vis-à-vis d'une politique de sécurité peut avoir un impact limité sur leur réussite en comparaison à une démarche de sensibilisation~\cite{Rhee09}~\cite{Sitnikova13}~\cite{Angafor20}. Dans ce cadre, ce Projet de Master vise à construire un démonstrateur d'attaque visant une partie d'un système industriel de contrôle d'une centrale nucléaire et plus précisément le \emph{Contrôle-Commande} (CC) du pont polaire. L'attaque présentée ici, appelée \emph{NinjaCrane}, infecte une station d'ingénierie grâce à un câble USB infecté ou une souris modifiée et déploie un logiciel malveillant visant la communication avec le \emph{Modicon M580}, un \emph{Automate Prgorammable Industriel} (API), en conduisant une attaque de type \emph{Homme Du Milieu} (HDM) exploitant les failles CVE-2022-45789 (une attaque par rejeu) et CVE-2021-22779 (appelée \emph{ModiPwn}). Ce travail illustre et souligne la necessité du respect de la politique de sécurité et s'inscrit dans une démarche de sensibilisation.

\iffalse
For a doctoral thesis, you have to provide a French translation of the
English abstract. For other projects this is optional.
\fi
\end{frenchabstract}

\cleardoublepage