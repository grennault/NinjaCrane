%%%%%%%%%%%%%%%%%%%%%%
\chapter{Related Work}
% The related work section covers closely related work. Here you can highlight the related work, how it solved the problem, and why it solved a different problem. Do not play down the importance of related work, all of these systems have been published and evaluated! Say what is different and how you overcome some of the weaknesses of related work by discussing the  trade-offs. Stay positive!

% This section is usually 3-5 pages.
%%%%%%%%%%%%%%%%%%%%%%

In this chapter is described the closely related work, without any attempt to compile an exhaustive list. 
In the MSc thesis was developed an APT attack on a non-simulated NPP-related process testbed. Most of the literature is based either on process simulation or on system virtualization. First section describes some cyberattack demonstrators or testbeds of ICS based on virtualization or on a hybrid configuration (called Hardware-In-The-Loop) of software-virtualization and hardware. They offer better flexibility, lower cost, better portability but suffer from a lack of representativeness compared to this study. Second section focuses on related work that includes non-simulated industrial process.

\section{Hardware-In-The-Loop-based or Virtualized Attack Demonstrator on ICS Based on Simulated Industrial Process}

De Brito et al.~\cite{deBrito22} built a Modbus/TCP network testbed simulating the NPP process (with Asherah NPP Simulator) and including a SCADA system (ScadaBR), a rogue PLC (OpenPLC on Arduino), an inside adversary (Kali Linux system), and a historian. The proposed cyberattack assumes that the adversary already neutralized a control system and replaced it by a rogue PLC. The rogue PLC maliciously tampers a pump speed variable while the Kali Linux system conducts a MITM attack so that the SCADA does not notice this tampering. De Brito et al. also explored a network monitoring defense based on the "Time from request".  This work proposes a low-cost, simplified, and hybrid network testbed to conduct cyberattack on an ICS of a NPP. As this Master thesis, a MITM attack on Modbus/TCP is conducted, however the adversary is assumed to be inside the network with a Kali Linux system and a rogue PLC and adversary directly targets the NPP process. 

Puys et al.~\cite{puys21} proposed two Hardware-In-The-Loop cyber-ranges in order to build an ICS cybersecurity awareness and a student training program respectively. First one, WonderICS, includes 3D simulation of an industrial process (i.e., hazardous gases management, hydroelectric power plant, chemical process of Tennessee-Eastman), PLCs, protection relay,  HMIs, Kali Linux Virtual Machine (VM), engineering workstation VM. Two attack scenarios were considered but the first one is the closest to the attack presented in this MSc project. It makes use of a Rubber Ducky USB key that performs cryptolocking, keyboard disabling, and PLC commands injection. The attack injects a malicious Modbus frame writing a fake value on the PLC's register. In comparison to the \emph{NinjaCrane} attack, it offers less stealthiness - a malicious cable or mouse is more stealth than a Rubber Ducky - and no flexibility - as the charge delivery and exploitation are automatically and instantaneously launching when Rubber Ducky is plugged-in. The second HIL cyber-range is called G-ICS and it offers a student training program. The students can then practice malicious packet injection, firewall rule editing (to protect against a known exploit), industrial proprietary protocols reverse-engineering and protocol fuzzing to find a protocol vulnerability.

Much research have been made to construct an ICS testbed for cybersecurity training such as the KYPO4INDUSTRY~\cite{celeda20}, the Micro-CI Testbed from Hurst et al.~\cite{ljmu5763}, or the EPS-ICS Testbed from Gao et al.\cite{Gao13}. None of them includes an APT scenario, however they all propose a flexible, pedagogical, portable and low-cost solution.

\section{Cybersecurity Testbed on ICS Based on Non-simulated Industrial Process}

Several universities have developed more realistic industrial processes such as Technische Hochschule Augsburg with LICSTER~\cite{Sauer19}. LICSTER is a simple open-source and low-cost ICS testbed where several single step attacks at any ICS level can be conducted such as sniffing, DoS or MITM. It benefits from its simplicity while still proposing a haptic understanding of the consequences of the cyberattack on the ICS. The testbed includes open-source Remote IO, PLC, SCADA and HMI. 

The New Orleans University SCADA testbed from Ahmed et al.~\cite{Ahmed16} includes three (simulated and non-simulated) industrial processes. It is made of PLC, Relay, switch, historian and HMI. Even though no cyberattack has been made it offers diversity, with the presence of three different vendors PLC to reverse-engineer the proprietary communication protocols.

A practical cyberattack demonstrator has also been made by \emph{Bristol Cyber Security Group}~\cite{Gardiner19}~\cite{Craggs19}~\cite{Rashid20}, where A. Rashid et al. created a full attack scenario starting from cloud IoT infrastructure and causing at the end water treatment process enter into an unsafe state. They exploited \emph{Tomcat} vulnerabilities, malicious PDF and a zero-day \emph{SCADAPack} vulnerability. This testbed illustrates how complex an attack in several steps can be made. The attacker is still assumed to be inside the SCADA network but A. Rashid et al. suggest that a malicious USB device could be the attack vector. This ICS training testbed including an APT scenario is believed to be the most advanced one and realist in the literature. In comparison, the \emph{NinjaCrane} attack can be seen as another APT scenario that reinforces the need to monitor the supply chain and the need to integrate cryptographic authentication standard in the industrial proprietary protocol. 

This need to have a better fidelity in the industry field pushed companies to build their own ICS testbed like the OPSWAT (CIP Lab) electric station ICS~\cite{opswat22} or the \emph{Check Point} chemical plant ICS testbed~\cite{checkpoint2023}. \emph{OPSWAT} (i.e. \emph{CIP Lab}) built an electric station ICS to promote their \emph{OTfuse} defense solution~\cite{opswat22}. The second testbed, the \emph{Check Point 1200R} security gateway is connected to the ICS network and prevents part of the fuzzing attacks, value-tampering attacks or exploitation of known vulnerabilities. This work is focusing on the ICS defense and not on the attack exploration nor the awareness part unlike presented work. As a side observation, the \emph{NinjaCrane} attack exploiting the \emph{ModiPwn} vulnerability would not have been detected by their security gateway.

